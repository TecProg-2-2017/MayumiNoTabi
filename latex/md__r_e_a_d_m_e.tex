Introducao ao Desenv de jogos -\/ 2017/1

O jogo Mayumi no Tabi é um jogo de plataforma onde os monstros são baseados em folclore japonês. A estrutura de fases é baseada no Megaman, temos vários monstros e cada monstro possui uma habilidade diferenciada. A personagem principal a qual o jogador controla possui um arco.

A história do jogo é inspirada pelo mito da Caixa de Pandora. Temos uma cidade amaldiçoada e a protagonista está à procura do Boss Final, que abriu a caixa.

Para se ganhar basta matar o boss final e perde-\/se quando a sua vida chega a zero.

A tecla 'A' atira a flecha, 'S' para chutar, 'Z' para zoom +, 'X' para zoom -\/, E\-S\-C para sair, 'N' para mostrar H\-P, 'M' para mostrar colisão. As tecladas direcionais são usadas para movimentar o personagem, sendo que a tecla direcional é usada para pular (é permitido double jump). Para o Editor de Fase basta usar os botões do mouse para adicionar e retirar um tile.

A tecla 'U', 'P', 'O', 'I' fazem nascer os monstros Porco, Mike, Banshee e Mask, respectivamente. A tecla 'N' faz mostrar H\-P e 'M' blocos de colisão.

Membros da Equipe\-:

Artistas\-: André Accioly Lima -\/ 12/0059908 Danilo José Bispo Galvão -\/ 12/0114852 Ivan de Oliveira Mello -\/ 12/0121352

Programadores\-: Cristiano Krug Brust -\/ 15/0008058 Hector Rocha Margittay -\/ 15/0036647 Raphael Soares Ramis -\/ 14/0160299

Músico\-: Rodrigo Roriz Teodoro -\/ 140161678

O programador Cristiano fez a refatoração do código para método de componentes em vez de classes, conforme mostrado em sala de aula. Além de implementação de animações de monstros e estrutura do jogo. O programador Hector foi responsável pelo Editor de Fases, pelo qual construímos o cenário, além de auxiliar na estrutura do jogo. O programador Raphael foi responsável também pela implementação de I\-A dos monstros, além de auxílio na estrutura do jogo como som. Todos os membros participaram em decisões gerais do jogo, como escolha de sprites, decisões de implementação, decisões de balanceamento de monstros, etc.

Para instalar basta rodar o makefile, não há nenhuma dependência além de ter S\-D\-L2 instalada.


\begin{DoxyItemize}
\item Screenshots\-:   
\item Vídeo\-: 
\end{DoxyItemize}